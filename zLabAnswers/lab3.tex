%%%%%%%%%%%%%%%%%%%%%%%%%%%%%%%%%%%%%%%%%
%
% CMPT 424
% Some Semester
% Lab/Assignment/Project X
%
%%%%%%%%%%%%%%%%%%%%%%%%%%%%%%%%%%%%%%%%%

%%%%%%%%%%%%%%%%%%%%%%%%%%%%%%%%%%%%%%%%%
% Short Sectioned Assignment
% LaTeX Template
% Version 1.0 (5/5/12)
%
% This template has been downloaded from: http://www.LaTeXTemplates.com
% Original author: % Frits Wenneker (http://www.howtotex.com)
% License: CC BY-NC-SA 3.0 (http://creativecommons.org/licenses/by-nc-sa/3.0/)
% Modified by Alan G. Labouseur  - alan@labouseur.com
%
%%%%%%%%%%%%%%%%%%%%%%%%%%%%%%%%%%%%%%%%%

%----------------------------------------------------------------------------------------
%	PACKAGES AND OTHER DOCUMENT CONFIGURATIONS
%----------------------------------------------------------------------------------------

\documentclass[letterpaper, 10pt,DIV=13]{scrartcl} 

\usepackage[T1]{fontenc} % Use 8-bit encoding that has 256 glyphs
\usepackage[english]{babel} % English language/hyphenation
\usepackage{amsmath,amsfonts,amsthm,xfrac} % Math packages
\usepackage{sectsty} % Allows customizing section commands
\usepackage{graphicx}
\usepackage[lined,linesnumbered,commentsnumbered]{algorithm2e}
\usepackage{listings}
\usepackage{parskip}
\usepackage{lastpage}

\allsectionsfont{\normalfont\scshape} % Make all section titles in default font and small caps.

\usepackage{fancyhdr} % Custom headers and footers
\pagestyle{fancyplain} % Makes all pages in the document conform to the custom headers and footers

\fancyhead{} % No page header - if you want one, create it in the same way as the footers below
\fancyfoot[L]{} % Empty left footer
\fancyfoot[C]{} % Empty center footer
\fancyfoot[R]{page \thepage\ of \pageref{LastPage}} % Page numbering for right footer

\renewcommand{\headrulewidth}{0pt} % Remove header underlines
\renewcommand{\footrulewidth}{0pt} % Remove footer underlines
\setlength{\headheight}{13.6pt} % Customize the height of the header

\numberwithin{equation}{section} % Number equations within sections (i.e. 1.1, 1.2, 2.1, 2.2 instead of 1, 2, 3, 4)
\numberwithin{figure}{section} % Number figures within sections (i.e. 1.1, 1.2, 2.1, 2.2 instead of 1, 2, 3, 4)
\numberwithin{table}{section} % Number tables within sections (i.e. 1.1, 1.2, 2.1, 2.2 instead of 1, 2, 3, 4)

\setlength\parindent{0pt} % Removes all indentation from paragraphs.

\binoppenalty=3000
\relpenalty=3000

%----------------------------------------------------------------------------------------
%	TITLE SECTION
%----------------------------------------------------------------------------------------

\newcommand{\horrule}[1]{\rule{\linewidth}{#1}} % Create horizontal rule command with 1 argument of height

\title{	
   \normalfont \normalsize 
   \textsc{CMPT 424 - Fall 2024 - Dr. Labouseur} \\[10pt] % Header stuff.
   \horrule{0.5pt} \\[0.25cm] 	% Top horizontal rule
   \huge Assignment Two  \\     	    % Assignment title
   \horrule{0.5pt} \\[0.25cm] 	% Bottom horizontal rule
}

\author{Nicholas Longo \\ \normalsize nicholas.longo2@marist.edu}

\date{\normalsize\today} 	% Today's date.

\begin{document}
\maketitle % Print the title

%----------------------------------------------------------------------------------------
%   start PROBLEM ONE
%----------------------------------------------------------------------------------------
\section{Lab 3}

\subsection{Explain the difference between internal and external fragmentation. }
External fragmentation is the total memory space that exists to satisfy a request, but it is not contiguous. This means that there could be enough memory left for a process overall, but if there is not a contiguous space for it then it cannot fit. For example there could be 4k of space, the process could be 3k, but the biggest gap of space available is 2k. This means it will not work. Internal fragmentation means allocated memory could be larger than the requested memory. The size difference in memory is internal to the partition. An example could be having a process of size .5k with 1k available. With internal fragmentation, you could allocate the whole 1k for the process, but half of it is wasted. Also Internal fragmentation has fixed size partitions, where external fragmentation has variable size partitions. In simpler terms, internal fragmentation happens when there is wasted space within an allocated piece of memory, and external fragmentation happens when there is enough space but it is not in a format that allows the process to fit. 



\subsection{Given five (5) memory partitions of 100KB, 500KB, 200KB, 300KB, and	600KB	(in that	 order),	how would optimal, first-fit, best-fit, and worst-fit algorithms	place processes of 212KB, 417KB, 112KB,  and 426 KB (in that order)? 
}
Setting it up in the form: 
	A: 100KB
	B: 500KB
	C: 200KB
	D: 300KB
	E: 600KB

First-fit allocates the first hole that is big enough. 
	212KB would be allocated to B because it is the first amount of space that fits it. 417KB would be allocated to E, because E no longer has enough space to fit it and the other partitions cannot fit it. The 112KB process would go in the B partition because the 212KB process did not take up all of B and more can fit in it. Lastly, the 426KB process would not fit in any partition because there is not enough space left over in the B and E partitions, which were the only ones that could fit it in the first place. 

Best-fit allocates the smallest hole that is big enough. It searches the entire list, and fills memory based on what would leave the smallest hole. 
	212KB would go to partition D because that can fit it and would leave the smallest hole. 417KB would go to B because it would fit it with the smallest hole left behind, making it a better choice than partition E. 112KB would go to C, for the same reasons as above. And lastly 426KB process would go to partition E. This allows all of the processes to fit, where this was not a possibility by using First-fit

Worst-fit allocates the largest hole. It searches the entire list and produces the largest leftover hole. 
	212KB would go to partition E because that would leave the most space. 417KB process would go to partition B because that leaves the largest hole, and partition E already has enough space taken up that it would not fit. 112KB would go to partition E. Because the space left over in E before this process goes in was 388, which is greater than all remaining spots, this partition will be used over the others. Lastly, the 426KB process does not have a partition big enough to fit it, so it will not be used. This is an example of external fragmentation, because overall there is enough space available, just not in the contiguous manner that would have been necessary. 




\end{document}
