%%%%%%%%%%%%%%%%%%%%%%%%%%%%%%%%%%%%%%%%%
%
% CMPT 424
% Some Semester
% Lab/Assignment/Project X
%
%%%%%%%%%%%%%%%%%%%%%%%%%%%%%%%%%%%%%%%%%

%%%%%%%%%%%%%%%%%%%%%%%%%%%%%%%%%%%%%%%%%
% Short Sectioned Assignment
% LaTeX Template
% Version 1.0 (5/5/12)
%
% This template has been downloaded from: http://www.LaTeXTemplates.com
% Original author: % Frits Wenneker (http://www.howtotex.com)
% License: CC BY-NC-SA 3.0 (http://creativecommons.org/licenses/by-nc-sa/3.0/)
% Modified by Alan G. Labouseur  - alan@labouseur.com
%
%%%%%%%%%%%%%%%%%%%%%%%%%%%%%%%%%%%%%%%%%

%----------------------------------------------------------------------------------------
%	PACKAGES AND OTHER DOCUMENT CONFIGURATIONS
%----------------------------------------------------------------------------------------

\documentclass[letterpaper, 10pt,DIV=13]{scrartcl} 

\usepackage[T1]{fontenc} % Use 8-bit encoding that has 256 glyphs
\usepackage[english]{babel} % English language/hyphenation
\usepackage{amsmath,amsfonts,amsthm,xfrac} % Math packages
\usepackage{sectsty} % Allows customizing section commands
\usepackage{graphicx}
\usepackage[lined,linesnumbered,commentsnumbered]{algorithm2e}
\usepackage{listings}
\usepackage{parskip}
\usepackage{lastpage}

\allsectionsfont{\normalfont\scshape} % Make all section titles in default font and small caps.

\usepackage{fancyhdr} % Custom headers and footers
\pagestyle{fancyplain} % Makes all pages in the document conform to the custom headers and footers

\fancyhead{} % No page header - if you want one, create it in the same way as the footers below
\fancyfoot[L]{} % Empty left footer
\fancyfoot[C]{} % Empty center footer
\fancyfoot[R]{page \thepage\ of \pageref{LastPage}} % Page numbering for right footer

\renewcommand{\headrulewidth}{0pt} % Remove header underlines
\renewcommand{\footrulewidth}{0pt} % Remove footer underlines
\setlength{\headheight}{13.6pt} % Customize the height of the header

\numberwithin{equation}{section} % Number equations within sections (i.e. 1.1, 1.2, 2.1, 2.2 instead of 1, 2, 3, 4)
\numberwithin{figure}{section} % Number figures within sections (i.e. 1.1, 1.2, 2.1, 2.2 instead of 1, 2, 3, 4)
\numberwithin{table}{section} % Number tables within sections (i.e. 1.1, 1.2, 2.1, 2.2 instead of 1, 2, 3, 4)

\setlength\parindent{0pt} % Removes all indentation from paragraphs.

\binoppenalty=3000
\relpenalty=3000

%----------------------------------------------------------------------------------------
%	TITLE SECTION
%----------------------------------------------------------------------------------------

\newcommand{\horrule}[1]{\rule{\linewidth}{#1}} % Create horizontal rule command with 1 argument of height

\title{	
   \normalfont \normalsize 
   \textsc{CMPT 424 - Fall 2024 - Dr. Labouseur} \\[10pt] % Header stuff.
   \horrule{0.5pt} \\[0.25cm] 	% Top horizontal rule
   \huge iProject 3  \\     	    % Assignment title
   \horrule{0.5pt} \\[0.25cm] 	% Bottom horizontal rule
}

\author{Nicholas Longo \\ \normalsize nicholas.longo2@marist.edu}

\date{\normalsize\today} 	% Today's date.

\begin{document}
\maketitle % Print the title

%----------------------------------------------------------------------------------------
%   start PROBLEM ONE
%----------------------------------------------------------------------------------------
\section{Lab 5}

\subsection{Consider the following set of processes, with the length of the CPU burst given in milliseconds}
\begin{table}[h!]
    \centering
    \begin{tabular}{|c|c|c|}
        \hline
        \textbf{Process} & \textbf{Burst Time (ms)} & \textbf{Priority} \\
        \hline
        P1 & 10 & 3 \\
        \hline
        P2 & 1 & 1 \\
        \hline
        P3 & 2 & 3 \\
        \hline
        P4 & 1 & 4 \\
        \hline
        P5 & 5 & 2 \\
        \hline
    \end{tabular}
    \caption{Processes with CPU Burst Time and Priority}
    \label{tab:processes}
\end{table}

The processes arrived in the order P1, P2, P3, P4, P5 all at time 0. 

\subsection{Draw four Gantt charts that illustrate the execution of these processes using the following scheduling algorithms: FCFS, SJF, nonpreemptive priority (a smaller priority number implies a higher priority) and RR (quantum 1)}

\begin{table}[h!]
    \centering
    \begin{tabular}{l|ccccccccccccccccccc}
         & 0 & 1 & 2 & 3 & 4 & 5 & 6 & 7 & 8 & 9 & 10 & 11 & 12 & 13 & 14 & 15 & 16 & 17 & 18 \\
        \hline
        \textbf{FCFS} & P1 & P1 & P1 & P1 & P1 & P1 & P1 & P1 & P1 & P1 & P2 & P3 & P3 & P4 & P5 & P5 & P5 & P5 & P5 \\
        \textbf{SJF} & P2 & P4 & P3 & P3 & P5 & P5 & P5 & P5 & P5 & P1 & P1 & P1 & P1 & P1 & P1 & P1 & P1 & P1 & P1 \\
        \textbf{NPPQ} & P2 & P5 & P5 & P5 & P5 & P5 & P3 & P3 & P1 & P1 & P1 & P1 & P1 & P1 & P1 & P1 & P1 & P1 & P4 \\
        \textbf{RR Q1} & P1 & P2 & P3 & P4 & P5 & P1 & P3 & P5 & P1 & P5 & P1 & P5 & P1 & P5 & P1 & P1 & P1 & P1 & P1 \\
    \end{tabular}
    \caption{Gantt charts for different scheduling algorithms}
    \label{tab:gantt_charts}
\end{table}

\subsection{What is the turnaround time of each process for each of the scheduling algorithms in part a?}

\begin{table}[h!]
    \centering
    \begin{tabular}{|c|c|c|c|c|}
        \hline
        \textbf{Process} & \textbf{FCFS} & \textbf{SJF} & \textbf{NPPQ} & \textbf{RR} \\
        \hline
        P1 & 10 & 19 & 18 & 19 \\
        \hline
        P2 & 11 & 1 & 1 & 2 \\
        \hline
        P3 & 13 & 4 & 8 & 6 \\
        \hline
        P4 & 14 & 2 & 19 & 4 \\
        \hline
        P5 & 19 & 9 & 6 & 14 \\
        \hline
    \end{tabular}
    \caption{Turnaround time of each process for each scheduling algorithm}
    \label{tab:turnaround_time}
\end{table}

\subsection{What is the waiting time of each process for each of these scheduling algorithms?}

\begin{table}[h!]
    \centering
    \begin{tabular}{|c|c|c|c|c|}
        \hline
        \textbf{Process} & \textbf{FCFS} & \textbf{SJF} & \textbf{NPPQ} & \textbf{RR} \\
        \hline
        P1 & 0 & 9 & 7 & 9 \\
        \hline
        P2 & 10 & 0 & 0 & 1 \\
        \hline
        P3 & 11 & 2 & 6 & 5 \\
        \hline
        P4 & 13 & 1 & 18 & 3 \\
        \hline
        P5 & 14 & 4 & 1 & 8 \\
        \hline
    \end{tabular}
    \caption{Waiting time of each process for each scheduling algorithm}
    \label{tab:waiting_time}
\end{table}

\subsection{Which of the algorithms results in the minimum average waiting time (over all processes)?}
SJF resulted in the minimum average waiting time. FCFS had 9.6, SJF had 3.2, NPPQ had 6.4, RR had 5.2





\end{document}
